% Options for packages loaded elsewhere
\PassOptionsToPackage{unicode}{hyperref}
\PassOptionsToPackage{hyphens}{url}
%
\documentclass[
]{article}
\usepackage{amsmath,amssymb}
\usepackage{lmodern}
\usepackage{iftex}
\ifPDFTeX
  \usepackage[T1]{fontenc}
  \usepackage[utf8]{inputenc}
  \usepackage{textcomp} % provide euro and other symbols
\else % if luatex or xetex
  \usepackage{unicode-math}
  \defaultfontfeatures{Scale=MatchLowercase}
  \defaultfontfeatures[\rmfamily]{Ligatures=TeX,Scale=1}
\fi
% Use upquote if available, for straight quotes in verbatim environments
\IfFileExists{upquote.sty}{\usepackage{upquote}}{}
\IfFileExists{microtype.sty}{% use microtype if available
  \usepackage[]{microtype}
  \UseMicrotypeSet[protrusion]{basicmath} % disable protrusion for tt fonts
}{}
\makeatletter
\@ifundefined{KOMAClassName}{% if non-KOMA class
  \IfFileExists{parskip.sty}{%
    \usepackage{parskip}
  }{% else
    \setlength{\parindent}{0pt}
    \setlength{\parskip}{6pt plus 2pt minus 1pt}}
}{% if KOMA class
  \KOMAoptions{parskip=half}}
\makeatother
\usepackage{xcolor}
\usepackage[margin=1in]{geometry}
\usepackage{graphicx}
\makeatletter
\def\maxwidth{\ifdim\Gin@nat@width>\linewidth\linewidth\else\Gin@nat@width\fi}
\def\maxheight{\ifdim\Gin@nat@height>\textheight\textheight\else\Gin@nat@height\fi}
\makeatother
% Scale images if necessary, so that they will not overflow the page
% margins by default, and it is still possible to overwrite the defaults
% using explicit options in \includegraphics[width, height, ...]{}
\setkeys{Gin}{width=\maxwidth,height=\maxheight,keepaspectratio}
% Set default figure placement to htbp
\makeatletter
\def\fps@figure{htbp}
\makeatother
\setlength{\emergencystretch}{3em} % prevent overfull lines
\providecommand{\tightlist}{%
  \setlength{\itemsep}{0pt}\setlength{\parskip}{0pt}}
\setcounter{secnumdepth}{-\maxdimen} % remove section numbering
\ifLuaTeX
  \usepackage{selnolig}  % disable illegal ligatures
\fi
\IfFileExists{bookmark.sty}{\usepackage{bookmark}}{\usepackage{hyperref}}
\IfFileExists{xurl.sty}{\usepackage{xurl}}{} % add URL line breaks if available
\urlstyle{same} % disable monospaced font for URLs
\hypersetup{
  pdftitle={Antimicrobial Resistance in Organic and Conventional farms: A sytemic literature review},
  hidelinks,
  pdfcreator={LaTeX via pandoc}}

\title{Antimicrobial Resistance in Organic and Conventional farms: A
sytemic literature review}
\author{}
\date{\vspace{-2.5em}September 09, 2022}

\begin{document}
\maketitle

We identified 1836 references in three search databases: Pub Med, Web of
Science, and PubAg, from studies published between 2000 and 2022. The
studies reported the point prevalence of antimicrobial resistance in
Organic and Conventional farms. Three duplicates were removed, and the
remaining 1833 unique references were subjected to relevance screening,
with 1744 references being removed. After assessing the remaining 89
references for eligibility, 17 were removed. The systematic review
included 72 studies that met our criteria (\emph{fig.1}). The North
America accounted for 46\% (n=33) of the surveys.Europe accounted for
36\% (n=26), 14\% (n=10) of studies were from Asia while Oceania and
South America contributed to 3\% (n=2) and 1\% (n=1) of the studies
respectively.

\includegraphics{AMR_manuscript_files/figure-latex/no_studies-1.pdf}

\hypertarget{antimicrobial-resistance-trends-in-organic-and-conventional-farms}{%
\paragraph{Antimicrobial resistance trends in Organic and conventional
farms}\label{antimicrobial-resistance-trends-in-organic-and-conventional-farms}}

Percentage resistance of antimicrobial compounds increased in both
organic and conventional farms between 2001 and 2020. The percentage
resistance in organic farms increased from 10\% 95\% Confidence interval
(CI: 2-18\%) to 32\% (CI: 23-41\%) while in conventional farms the
percentage resistance increased from 18\% (CI: 12-28\%) to 42\% (CI:
33-51\%) (p = 0.0215) (\emph{fig2}). Generally, the mean resistance is
higher in conventional farms 28\% as compared to organic farms 18\%
(supplementary materials).

\includegraphics{AMR_manuscript_files/figure-latex/trends-1.pdf}

\hypertarget{host-resistance}{%
\paragraph{Host resistance}\label{host-resistance}}

In conventional farms, cattle, chicken, pigs, turkeys, and environments
accounted for 14, 27, 16, 1 and 6 studies respectively while in organic
farms, cattle, chicken, pigs, turkeys and environments accounted for 15,
32, 13, 2 and 5 studies respectively. There was higher resistance in
conventional farms as compared to organic farms in cattle, chicken,
pigs, and turkeys except in the environment where there was higher
resistance in organic farms as compared to the conventional farms. In
conventional farms, the resistance was 14.5\% in cattle, 22\% in
chicken, 11.5\% in environment, 24.5\% in pigs and 46\% in turkeys while
the resistance rate in organic farms was 9\% in cattle, 13.5\% in
chicken, 16\% in environment, 15\% in pigs and 22.5\% in turkeys.

Figure 2

\hypertarget{foodborne-pathogen-resistance}{%
\paragraph{Foodborne pathogen
resistance}\label{foodborne-pathogen-resistance}}

Most of the surveys covered antimicrobials classified as critically
important for human medicine by the World Health Organization (WHO,
2019). Our study covered 350,570 isolates from organic and conventional
farms in Asia, Oceania, Europe, South America and North America testing
resistance of Escherichia coli, Salmonella, Campylobacter, and
Staphylococcus aureus.

The resistance in E. coli isolates in conventional farms was higher than
in organic farms. There was higher resistance of amoxicillin-clavulanic
acid (AMC) in Asia, 98\%, (CI:95-100\%) in both conventional and organic
farms as compared to 32\%, (CI:29-35\%) and 20\%, (CI:18-22\%) in
European conventional and organic farms respectively. In South America,
the resistance of E. coli against AMC was 23\%, (CI:16-30\%) in organic
farms while the resistance was lower in the North American organic farms
at 3\%. North America reported higher resistances in erythromycin 76\%,
(CI:73-79\%) in conventional farms and 56\%, (CI: 53-59\%) in organic
farms. Tylosin resistance was 64\%, (CI:60-68\%) and 36\%, (CI: 32-40\%)
in conventional and organic farms respectively. Neomycin resistance was
63\%, (CI:59-67\%) in conventional and 24\%, (CI:21-27\%) in organic
farms in North America while the resistance rate was 2\%, (CI: 1-3\%) in
organic farms in Oceania.

The resistance was higher in salmonella in Asia as compared to North
America. Resistance of ampicillin in salmonella in Asia was 52\%,
(CI:47-57\%) and 23\%, (CI: 19-27\%) in conventional and organic farms
respectively while in North America the resistance was 22\%, (CI:
21-23\%) and 18\%, (CI: 16-20\%) in conventional and organic farms
respectively. The salmonella resistance against streptomycin was 46\%,
(CI: 40-52\%) and 21\%, (CI: 17-25\%) in conventional and organic farms
respectively in Asia, while in North America, the streptomycin
resistance against salmonella was 20\%, (CI:19-21\%) and 10\%, (CI:
9-11\%) in conventional and organic farms respectively. Gentamycin
resistance was higher in conventional farms 31\%, (CI: 27-35\%) as
compared to organic farms 4\%, (CI: 2-6\%) in Asia while in North
America there was no significant difference between resistances in
organic farms, 3\%, (CI: 2-4\%) and conventional farms 4\%, (CI:3-5\%).
This was the same case in amoxicillin-clavulanic acid in North America
where resistance in conventional farms was 23\%, (CI: 21-25\%) and 24\%,
(CI: 22-26\%). However, there was a significant difference in
amoxicillin-clavulanic acid resistance in Asia where organic farms had
6\%, (CI: 3-9\%) as compared to 19\%, (CI: 15-23\%).

In Europe, campylobacter had a higher resistance in norfloxacin,
ofloxacin and ampicillin in organic farms with resistance rates of 91\%
(CI: 82-100\%), 91\% (CI: 82-100\%) and 79\% (CI: 67-91\%) respectively.
However, ampicillin recorded lower resistance rates than in conventional
farms 66\% (CI: 51-81\%) as compared to organic farms. Ciprofloxacin
resistance was lower in North America, 2\% in organic farms as compared
to 11\% (CI: 10-12\%) in organic and conventional farms respectively as
compared to Europe 30\% (CI: 27-33\%) and 36\% (32-40\%) in organic and
conventional farms respectively. Nalidixic acid resistance was higher in
Europe as compared to North America in conventional and organic farms
with resistance percentages of 31\% (CI: 27-35\%) and 35\% (CI: 31-39\%)
in organic and conventional farms respectively as compared to 5\% (CI:
4-6\%) and 12\% (CI: 11-13\%) in organic and conventional farms
respectively in North America.

Europe, North America, and Asia recorded the same resistance of
erythromycin in enterococcus, 17\% in organic farms. However, the
resistance was higher in conventional farms as compared to organic
farms. Oceania had a higher resistance of ciprofloxacin in conventional
farms at 84\% (CI: 80-88\%) as compared to Asia, North America and
Europe which recorded 31\% (CI: 18-44\%), 18\% (13-23\%) and 11\% (CI:
7-15\%) respectively. Europe however had higher resistance against
Staphylococcus aureus in conventional farms 75\% (CI: 67-83\%) as
compared to 38\% (CI: 29-47\%) in organic farms. Overall, the
erythromycin resistance was lower in organic farms in Europe and North
America with resistances of 10\% (CI: 7-13\%) and 6\% (CI: 3-9\%)
respectively.

\includegraphics{AMR_manuscript_files/figure-latex/continents-1.pdf}

\begin{verbatim}
## TableGrob (2 x 2) "arrange": 3 grobs
##   z     cells    name                 grob
## 1 1 (1-1,2-2) arrange       gtable[layout]
## 2 2 (2-2,2-2) arrange text[GRID.text.2021]
## 3 3 (1-2,1-1) arrange text[GRID.text.2022]
\end{verbatim}

\hypertarget{antimicrobial-class-resistance-at-the-country-level}{%
\paragraph{Antimicrobial class resistance at the country
level}\label{antimicrobial-class-resistance-at-the-country-level}}

China had a higher resistance to macrolide, a critically important
antimicrobial to human medicine in organic and conventional farms at
97\% (CI: 94-100\%) and 100\% respectively as compared to the United
States of America which recorded 21\% CI (20-11\%) and 33\% CI:
(32-34\%) in organic and conventional farms respectively. Tetracycline
antibiotic class had lower resistance rates in both organic and
conventional farms by 15\% (CI: 14-16\%) in Sweden as compared to the
United States of America with 55\% (CI: 54-56\%) and 38\% (CI: 37-39\%)
in conventional and organic farms respectively. China had the highest
sulfonamide resistance in conventional farms, 31\% (25-37\%) and the
lowest sulfonamide resistance in organic farms, 2\% CI: (0-4\%) as
compared to other countries while both USA and Sweden reported the
lowest resistance rate in organic farms at 11\%.

\hypertarget{drug-acronyms}{%
\paragraph{Drug acronyms}\label{drug-acronyms}}

\textbf{To be added when maps are ready}

\end{document}
